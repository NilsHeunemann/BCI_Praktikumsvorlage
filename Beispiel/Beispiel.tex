\documentclass[a4paper, 12pt]{article}
\usepackage[utf8]{inputenc}

% Deutsche Spracheinstellungen; im zweifelsfall ändern
\usepackage[ngerman]{babel}

\usepackage{amsmath}

\usepackage[
 math-style=ISO,
 bold-style=ISO,
 sans-style=italic,
 nabla=upright,
 partial=upright,
]{unicode-math}
\setmathfont{xits-math} %uncomment for Arial

\usepackage{fontspec}
\usepackage[T1]{fontenc}
\setmainfont{xits} % Times New Roman -> 12pt
%\setmainfont{Liberation Sans} % Arial -> 11pt
 
% Quellen
\usepackage[sorting=none]{biblatex}
\addbibresource{lit.bib}

% Römische Nummerierung für Abschnitte
%\renewcommand{\thesection}{\Roman{section}} 
%\renewcommand{\thesubsection}{\thesection.\Roman{subsection}}


% 1,5-fachen Zeilenabstand festlegen
\usepackage[onehalfspacing]{setspace}

% Seitenränder einstellen
\usepackage[left=2.5cm, right=2.5cm, top=2cm, bottom=2.5cm]{geometry}

% richtige Anführungszeichen
\usepackage[autostyle]{csquotes}

% Klick bare Links im Dokument
\usepackage{hyperref}
\hypersetup{
    colorlinks=true,
    linkcolor=black,
    filecolor=magenta,      
    citecolor=black,
    urlcolor=blue,
    pdftitle={},
}
    
% Keine Einrückung, der Befehl \indent funktioniert aber trotzdem
\newlength\tindent
\setlength{\tindent}{\parindent}
\setlength{\parindent}{0pt}
\renewcommand{\indent}{\hspace*{\tindent}}

% Grafikfähigkeit
\usepackage{graphics}
\usepackage{wrapfig}

% Für schönere Einheiten
\usepackage[
 locale=DE,
 separate-uncertainty=true,
]{siunitx}

% Noch ein wenig Chemie
\usepackage[version=4]{mhchem}

% Definieren des reaction Environments für nummerierte Reaktionen
\newcounter{reaction}
\newcounter{tmp}
\newenvironment{reaction}{%
\setcounter{tmp}{\value{equation}}%
\setcounter{equation}{\value{reaction}}%
\renewcommand{\theequation}{\Roman{equation}}%
\begin{equation}%
}{%
\end{equation}%
\setcounter{reaction}{\value{equation}}%
\setcounter{equation}{\value{tmp}}%
}

\usepackage{pdflscape}
\usepackage[format=hang]{caption}

\usepackage[upright=true]{derivative}
\RenewDerivative\pdv{\symup{\partial}}[delims-eval={. \rver}]

\usepackage[section]{placeins}

\usepackage{booktabs}

\usepackage{tocloft}
\addtolength{\cftsubsecnumwidth}{10pt}

\addbibresource{Beispiel/beispiel_lit.bib}

\begin{document}

\tableofcontents

\newpage

\section{Gestaltungsrichtlinien}

In diesem Dokument sollen die Gestaltungsrichtlinien aus \textcquote{wie_protokoll}{Wie schreibe ich ein Protokoll?} umgesetzt werden.

\subsection{Layout}

\begin{itemize}
	\item Schriftart: Times New Roman (Liberation Serif) oder Arial (Liberation Sans)
	\item Schriftgröße: 12pt. (Times New Roman), 11pt. (Arial)
	\item Zeilenabstand: 1,5-fach (hier tatsächlich richtig umgesetzt)
	\item Blocksatzt (standard)
	\item Einheiten im Text nicht von der Zahl trennen und mit richtigen Abständen mit dem Package \texttt{SIUnitx}

	\begin{itemize}
		\item Wert:\begin{verbatim}
			\qty{2}{\kilo\joule\per\kilo\gram\kelvin}
		\end{verbatim} 
		ergibt: \qty{2}{\kilo\joule\per\kilo\gram\kelvin} 	
		\item Einheiten: \begin{verbatim}
			\unit{\kilo\gram\per\cubic\metre}
		\end{verbatim}
		ergibt: \unit{\kilo\gram\per\cubic\metre}
	\end{itemize}
	
\end{itemize}

\subsection{Abbildungen}

Abbildungen immer mit Unterschriften versehen, wie in Abbildung \ref{fig:example_a} dargestellt ist. 
Bei Achsenbeschriftungen darauf achten, dass die Einheiten mit \(\mathbin{/}\) \textquote{weg dividiert werden}.
Mehr dazu in dem Abschnitt \ref{sec:typ} 

% Diagramm
\begin{figure}[h] 
    \centering
    \scalebox{.7}{\includegraphics{example-image-a}} % Größe anpassen
    \caption{Das Bild zeigt ein Beispielbild. Anstelle dieses Bildes könnte hier aber auch ein Diagramm oder eine andere Grafik sein.}
    \label{fig:example_a} % Nummer anpassen
\end{figure}

\FloatBarrier

\subsection{Tabellen}
Tabellen bekommen nur Überschriften wie in Tabelle \ref{tab:use} zu sehen ist.

\begin{table}[h]
	\centering
	\caption{Wichtige Überschrift}
	\begin{tabular}{c|c c}
		Wert & Wert & Wert \\
		\hline
		\num{12} &  \num{12,17}& \num{12,345} \\
		\hline		
	\end{tabular}
	\label{tab:use}	
\end{table}

Das Package \texttt{SIUnitx} \cite{SIUnitx} erlaubt es schöntere Tablellen mit Werten zu erstellen. Die Tablelle \ref{tab:si} ist ein Beispiel dafür.

\begin{table}[h]
	\centering
	\caption{Tabelle mit \texttt{SIUnitx}}
	\sisetup{table-format=2.1} %Stellen vor & nach dem Komma
	\begin{tabular}{S S}
		\toprule
		{$k_0 \mathbin{/} \unit{\litre\per\mole\per\second}$} & {$E_A \mathbin{/} \unit{\kilo\joule\per\mole}$} \\
		\midrule
		\num{1,81 e7} & \num{48,3} \\
		\bottomrule
	\end{tabular}
	\label{tab:si}	
\end{table}

\subsection{Chemische Reaktionen}

Chemische Formeln oder Elemente werden aufrecht geschrieben wie beispielsweise \ce{Br2}.
Reaktionen werden mit Römischen Zahlen nummeriert wie Reaktion \eqref{re:ex}.

% nummerierte Chemische Reaktion
\begin{reaction} \label{re:ex}
	\ce{Br2 ->[h_\nu] 2Br.} % Hier die Reaktion anpassen
\end{reaction}

	
\section{Typographie}\label{sec:typ}
Die Hinweise hier sind nach meinem besten Wissen verfasst, über Korrekturen freue ich mich immer.

Variablen werden sowohl im Text als auch in Formeln kursiv gesetzt werden.
Das macht \LaTeX{} glücklicherweise von alleine, wenn man dazu die richtigen Umgebungen nutzt.
Einheiten werden als mathematische Zeichen betrachtet und können daher nach den Regeln der Algebra behandelt werden.
Daher werden die einheitsbehafteten Größen bei Tabellen und Diagrammen auch durch die entsprechende Einheit geteilt \cite{si_brochure}.
Es ist also möglich \(p = \qty{48}{\kilo\pascal}\) auch als \(p\mathbin{/}\unit{\kilo\pascal} = 48 \) zu schreiben.	
Die Gleichung \eqref{eq:dampf} zeigt ein anschaulicheres Beispiel.

\begin{equation}\label{eq:dampf}
	p \mathbin{/} \unit{\kilo\pascal} = A - \frac{B}{T \mathbin{/} \unit{\kelvin} + C}
\end{equation}

Des weiteren sollte darauf geachtet werden, dass mathematische Operatoren und Konstanten nicht kursiv zu setzen sind \cite{green_book}. 
Physikalische Konstanten jedoch sind kursiv zu setzen Beispiele sind die Gleichung \eqref{eq:int1} und \eqref{eq:int2}.
Dabei fällt besonders bei dem Dell, dem Delta so wie bei dem d des Integrals auf, dass diese alle aufrecht gesetzt sind.

\begin{align}
	\pdv[order={1}]{\ln K}{T} &= \frac{\symup{\Delta}H_r}{RT^2} \label{eq:int1} \\
	\Leftrightarrow K(T) &= \int_{T_0}^T \frac{\symup{\Delta}H_\text{r}}{RT^2} \odif{T} \label{eq:int2} \\
	\intertext{Indizes sind auch aufrecht zu setzen wie in Gleichung \eqref{eq:indx}}
	\Dot{m} _{\text{akk.}} &= \Dot{m}_{\text{ein}} - \Dot{m} _{\text{aus}} \label{eq:indx}   
\end{align}


\newpage

\printbibliography
\end{document}
